\documentclass{article}
\usepackage{xurl}




\begin{document}

\section{Introduction}

\subsection{Introduction}

\begin{itemize}
  \item Aquí se ven los tipeados que se van a utilizar durante todo el proceso del libro.
  \item Dónde encontrar ayuda cuando se presenta un error (foros de comunidad).
\end{itemize}

\section{Preparing for the build}

\subsection{Preparing the Host System}

\begin{itemize}
  \item En la parte de \textit{Host System Requirements} se corrió un \textit{script} en \textit{bash} que permite ver cómo están los paquetes dentro del ecosistema de nuestra máquina virtual. Para nuestro caso faltó instalar Bison, GCC, texalgo y que me acuerde que se apunte bison a yacc.
  \item Se han creado las particiones:
        \begin{itemize}
          \item Una partición de 30.27 GB en btrfs para todo lo que es el sistema.
          \item Otra de \textit{swap} del tamaño de la memoria RAM de la máquina virtual (8.11 GB).
          \item Se tiene 700 MB de espacio libre para montar el GRUB en el futuro. Aunque es necesario solo 1 MB
          \item La partición de btrfs se formateó a ext4 debido a lectura que se hizo. Esto se debe a que ext4 fue implementado desde antes en el Linux kernel y el btrfs desde la versión 5.0, pero pero pero, en el libro se utiliza una cuatro punto algo.
          \item Se ha creado el sistema de archivos en la partición hecha y también sobre la partición \textit{swap}, no arrojó ningún mensaje de error. Todo bien.
          \item Se ha creado la variable \texttt{LFS} y se editó el archivo \texttt{.bashrc} de ambos usuarios (vm y root) donde se ha agregó unas líneas en bash para comprobar que la variable existe y si no, que la cree.
          \item Se ha montado en la partición el sistema de archivos con el comando \texttt{mount} y también para la memoria \textit{swap}.
        \end{itemize}
\end{itemize}

\subsection{Packages and Patches}

\begin{itemize}
  \item El paquete \url{https://prdownloads.sourceforge.net/expat/expat-2.6.2.tar.xz} no está disponible, arroja un error 404, tocó bajarlo de un repositorio en GitHub. Del resto todo bien, recordar utilizar \texttt{sudo} para la parte de descargar y no tener que luego hacer permisos de \textit{root}.
  \item Se pasó el archivo \texttt{md5sums} para validar todos los paquetes.
  \item Hubo un problema de que algunos paquetes no los descargó, esto porque se tenía la fecha desicronizada, toca siempre que se vuelva a utilizar la VM verificar la fecha.
  \item Se hizo las actualizaciones de los paquetes por temas de seguridad.
\end{itemize}

\end{document}
